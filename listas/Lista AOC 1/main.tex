\documentclass{article}
\usepackage{xcolor}
\usepackage[a4paper, total={6in, 8in}]{geometry}
\usepackage{enumitem}
\usepackage{listings}
\usepackage{float}
\usepackage{setspace}

\title{\Huge Lista 1}
\author{\Large Arquitetura e Organização de Computadores}
\date{Março 2024}

\begin{document}

\maketitle

\begin{enumerate}

\item \large Dentre as alternativas, qual programa em MIPS melhor traduz o código em C abaixo?

\begin{center}
    \begin{minipage}{0.5\textwidth}
        \begin{lstlisting}[frame=single]
    int abs(int num) {
        if(num < 0)
            return 0 - num;
        else
            return num;
    }
        \end{lstlisting}
    \end{minipage}
\end{center}

\begin{minipage}{0.4\textwidth}
    \begin{lstlisting}[frame=single]
a)  abs:
    bge $a0, $0, else
    sub $v0, $0, $a0
    j abs_exit
    else:
    add $v0, $0, $a0
    abs_exit:
    jr $ra
    \end{lstlisting}%certo
\end{minipage}
\hspace{1cm}
\begin{minipage}{0.4\textwidth}
    \begin{lstlisting}[frame=single]
b)  abs:
    blt $a0, $0, else
    sub $v0, $0, $a0
    j abs_exit
    else:
    add $v0, $0, $a0
    abs_exit:
    jr $ra
    \end{lstlisting}
\end{minipage}

\bigskip

\begin{minipage}{0.4\textwidth}
    \begin{lstlisting}[frame=single]
c)  abs:
    bge $a0, $0, then
    add $v0, $0, $a0
    j abs_exit
    then:
    sub $v0, $0, $a0
    abs_exit:
    jr $ra
    \end{lstlisting}
\end{minipage}
\hspace{1cm}
\begin{minipage}{0.4\textwidth}
    \begin{lstlisting}[frame=single]
d)  abs:
    blt $a0, $s0, else
    sub $v0, $s0, $a0
    j abs_exit
    else:
    add $v0, $s0, $a0
    abs_exit:
    jr $ra
    \end{lstlisting}
\end{minipage}

\pagebreak
\item \large Dentre as alternativas, quais programas em MIPS traduzem o código em C abaixo?

\begin{center}
    \begin{minipage}{0.5\textwidth}
        \begin{lstlisting}[frame=single]
    int min(int a, int b) {
        if(a < b)
            return a;
        else
            return b;
    }
        \end{lstlisting}
    \end{minipage}
\end{center}

\begin{minipage}{0.4\textwidth}
    \begin{lstlisting}[frame=single]
a)  min:
    blt $a0, $a1, then
    add $v0, $0, $a0
    j min_exit
    then:
    add $v0, $0, $a1
    min_exit:
    jr $ra
    \end{lstlisting}
\end{minipage}
\hspace{1cm}
\begin{minipage}{0.4\textwidth}
    \begin{lstlisting}[frame=single]
b)  min:
    add $v0, $0, $a0
    blt $a0, $a1, min_xt
    add $v0, $0, $a1
    min_xt:
    jr $ra
    \end{lstlisting}%certo
\end{minipage}

\bigskip

\begin{minipage}{0.4\textwidth}
    \begin{lstlisting}[frame=single]
c)  min:
    blt $a0, $a1, then
    add $v0, $0, $a1
    j min_exit
    then:
    add $v0, $0, $a0
    min_exit:
    jr $ra
    \end{lstlisting}%certo
\end{minipage}
\hspace{1cm}
\begin{minipage}{0.4\textwidth}
    \begin{lstlisting}[frame=single]
d)  min:
    bge $a0, $a1, then
    add $v0, $0, $a1
    j min_exit
    then:
    add $v0, $0, $a0
    min_exit:
    jr $ra
    \end{lstlisting}
\end{minipage}

\pagebreak

\item \large Após a execução do programa abaixo, qual o estado do trecho de memória e dos registradores acessados?

\begin{center}
\begin{minipage}{0.4\textwidth}
\begin{lstlisting}[frame=single]
lui $t0, 0x1001
addi $t0, $t0, 0x0008
addi $t1, $0, 0x01
sw $t1, 0($t0)
addi $t1, $0, 0x02
sw $t1, 4($t0)
addi $t1, $0, 0x0E
sw $t1, 8($t0)
addi $t1, $0, 0x10
sw $t1, 12($t0)
addi $t0, $t0, 0x0C
lw $t2, -4($t0)
lw $t3, -8($t0)
lw $t4, -12($t0)
\end{lstlisting}
\end{minipage}
\end{center}

\begin{centering}
\makebox[0pt][c]{
\begin{minipage}{0.55\textwidth}
a)
\begin{table}[H]
    \begin{tabular}{|l|l|l|l|}
    \hline
    memória    & valor & registrador & valor      \\ \hline
    0x10010000 & 0x00  & \$a2        & 0x00000000 \\ \hline
    0x10010004 & 0x00  & \$a3        & 0x00000000 \\ \hline
    0x10010008 & 0x01  & \$t0        & 0x10010014 \\ \hline
    0x1001000C & 0x02  & \$t1        & 0x00000001 \\ \hline
    0x10010010 & 0x0E  & \$t2        & 0x00000002 \\ \hline
    0x10010014 & 0x10  & \$t3        & 0x0000000E \\ \hline
    0x10010018 & 0x00  & \$t4        & 0x00000010 \\ \hline
    0x1001001C & 0x00  & \$t5        & 0x00000000 \\ \hline
    \end{tabular}
\end{table}
\end{minipage}
\begin{minipage}{0.55\textwidth}
b)
\begin{table}[H]
    \begin{tabular}{|l|l|l|l|}
    \hline
    memória    & valor & registrador & valor      \\ \hline
    0x10010000 & 0x00  & \$a2        & 0x00000000 \\ \hline
    0x10010004 & 0x00  & \$a3        & 0x00000000 \\ \hline
    0x10010008 & 0x01  & \$t0        & 0x10010014 \\ \hline
    0x1001000C & 0x02  & \$t1        & 0x00000010 \\ \hline
    0x10010010 & 0x01  & \$t2        & 0x0000000E \\ \hline
    0x10010014 & 0x02  & \$t3        & 0x00000002 \\ \hline
    0x10010018 & 0x0E  & \$t4        & 0x00000001 \\ \hline
    0x1001001C & 0x10  & \$t5        & 0x00000000 \\ \hline
    \end{tabular}
\end{table}
\end{minipage}
}

\bigskip

\makebox[0pt][c]{
\begin{minipage}{0.55\textwidth}
c)
\begin{table}[H]
    \begin{tabular}{|l|l|l|l|}
    \hline
    memória    & valor & registrador & valor      \\ \hline
    0x10010000 & 0x00  & \$a2        & 0x00000000 \\ \hline
    0x10010004 & 0x00  & \$a3        & 0x00000000 \\ \hline
    0x10010008 & 0x01  & \$t0        & 0x10010014 \\ \hline
    0x1001000C & 0x02  & \$t1        & 0x00000001 \\ \hline
    0x10010010 & 0x01  & \$t2        & 0x00000002 \\ \hline
    0x10010014 & 0x02  & \$t3        & 0x0000000E \\ \hline
    0x10010018 & 0x0E  & \$t4        & 0x00000010 \\ \hline
    0x1001001C & 0x10  & \$t5        & 0x00000000 \\ \hline
    \end{tabular}
\end{table}
\end{minipage}
\begin{minipage}{0.55\textwidth}
d)
\begin{table}[H]
    \begin{tabular}{|l|l|l|l|}
    \hline
    memória    & valor & registrador & valor      \\ \hline
    0x10010000 & 0x00  & \$a2        & 0x00000000 \\ \hline
    0x10010004 & 0x00  & \$a3        & 0x00000000 \\ \hline
    0x10010008 & 0x01  & \$t0        & 0x10010014 \\ \hline
    0x1001000C & 0x02  & \$t1        & 0x00000010 \\ \hline
    0x10010010 & 0x0E  & \$t2        & 0x0000000E \\ \hline
    0x10010014 & 0x10  & \$t3        & 0x00000002 \\ \hline
    0x10010018 & 0x00  & \$t4        & 0x00000001 \\ \hline
    0x1001001C & 0x00  & \$t5        & 0x00000000 \\ \hline
    \end{tabular}%certo
\end{table}
\end{minipage}
}

\end{centering}
\break

\item \large Considere que \verb|0xABCD|, \verb|0b1011_1010_1100_0001| e \verb|0b0100010100111110|  estão armazenados em big-endian. Como é sua representação em little-endian?

\begin{spacing}{1.5}
\begin{enumerate}
\item \verb|0xB3D5|, \verb|0b1000_0011_0101_1101| e \verb|0b0111110010100010|

\item \verb|0xCDAB|, \verb|0b1100_0001_1011_1010| e \verb|0b0011111001000101|%certo

\item \verb|0xDCBA|, \verb|0b0001_1100_1010_1011| e \verb|0b1110001101010100|
\end{enumerate}
\end{spacing}
\bigskip

\item \large Para cada instrução a seguir, indique sua representação em código de máquina MIPS:

\begin{spacing}{1.5}
\begin{enumerate}
\item \verb|ADD $s0, $s1, $s2|

\begin{enumerate}
\item \verb|0x02328020|%certo
\item \verb|0x02328022|
\item \verb|0x012A4020|
\end{enumerate}

\item \verb|J ident| (considere que \verb|ident| está no endereço 0x00400004)

\begin{enumerate}
\item \verb|0x00400018|
\item \verb|0x08100001|%certo
\item \verb|0x02000008|
\end{enumerate}

\item \verb|ADDI $t0, $t1, 1|

\begin{enumerate}
\item \verb|0x25280001|
\item \verb|0x22300001|
\item \verb|0x21280001|%certo
\end{enumerate}

\end{enumerate}
\end{spacing}
\bigskip

\break

\item \large Para cada código de máquina a seguir, indique a correspondente instrução em MIPS:

\begin{spacing}{1.5}
\begin{enumerate}
\item \verb|0x02328022|

\begin{enumerate}
\item \verb|and $s1, $s2, $s0|
\item \verb|add $s2, $s1, $s0|
\item \verb|sub $s0, $s1, $s2|%certo
\end{enumerate}

\item \verb|0x02000008|

\begin{enumerate}
\item \verb|jalr $s0|
\item \verb|jr $s0|%certo
\item \verb|jalr $s1|
\end{enumerate}

\item \verb|0x25280001|

\begin{enumerate}
\item \verb|addiu $t0, $t1, 1|%certo
\item \verb|addi $s0, $s1, 1|
\item \verb|addi $t1, $t0, 1|
\end{enumerate}

\end{enumerate}
\end{spacing}
\bigskip

Para as questões sobre representação em ponto fixo, considere a notação \\
\verb|fixo(c, p)|: valor armazenado em \verb|c| bits com ponto entre o bit \verb|p| e o bit \verb|p+1|.

\item \large Qual número é armazenado como \verb|0xA3| em \verb|fixo(8, 3)|?

\smallskip

\begin{enumerate}
\item \verb|-2.7360|
\item \verb|-5.8125|%certo
\item \verb|-9.5145|
\end{enumerate}

\bigskip

\item \large Qual valor hexadecimal representa o número \verb|-2.5| em \verb|fixo(8, 3)|?

\smallskip

\begin{enumerate}
\item \verb|0xD9|
\item \verb|0xEA|
\item \verb|0xD8|%certo
\end{enumerate}

\bigskip

\item \large Qual valor hexadecimal representa o número \verb|-2.5| em \verb|fixo(8, 0)|?

\smallskip

\begin{enumerate}
\item \verb|0x8E|
\item \verb|0xCD|
\item \verb|0xFB|%certo
\end{enumerate}

\bigskip

Exercício complementar:

\item \large Compile o código abaixo para MIPS. Indique os endereços e registradores correspondentes a cada variável.

\begin{lstlisting}[frame=single]
void swap(int *a, int *b) {
    int temp = *a;
    *a = *b;
    *b = temp;
}

int bubbleSort(int array[], int size) {

    int swaps = 0;

    for (int step = 0; step < size - 1; ++step) {
        for (int i = 0; i < size - step - 1; ++i) {
            if (array[i] > array[i + 1]) {
                swap(array + i, array + i + 1);
                swaps ++;
            }
        }
    }

    return swaps;
}

int main() {
    int a[] = {2, 1, 7, 4, 3, 9, 8, 6};
    int tam = sizeof(a)/sizeof(*a);
    int s = bubbleSort(a, tam);
    return 0;
}

\end{lstlisting}

\end{enumerate}

\end{document}
