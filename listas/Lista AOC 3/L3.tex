\documentclass{article}
\usepackage{xcolor}
\usepackage[a4paper, total={7in, 10in}]{geometry}
\usepackage{enumitem}
\usepackage{listings}
\usepackage{float}
\usepackage{setspace}
\usepackage{textcomp}
\usepackage{amsmath}
\usepackage{mathtools}

\title{\Huge Lista 3}
\author{\Large Arquitetura e Organização de Computadores}
\date{Junho de 2024}

\begin{document}
\large

\maketitle

\section*{Tema: memória cache}

Para as questões 1-6, considere uma cache, até então sem dados validos, com 16 conjuntos, 2 vias por conjunto, 8 palavras por via, palavras e endereços de 16 bits, write-back como estratégia de escrita e \textit{Menos Recentemente Usada} como política de substituição.

\begin{enumerate}

\item O campo de índice tem quantos bits nesta cache?

\item O campo de tag tem quantos bits nesta cache?

\item Qual o tamanho, em bits, da cache?

Para as questões 4 a 6, considere que o processador acessou, nesta ordem, os endereços \verb|0x00F0(L)|, \verb|0x00F1(L)|, \verb|0x0010(E)|, \verb|0x1018(L)|, \verb|0x2018(E)|, \verb|0x3018(E)|, \verb|0xFFF0(L)|, \verb|0xFF00(L)|. L \textrightarrow Leitura, E \textrightarrow Escrita.

\item Quantas requisições foram feitas à memória principal?

\item Quais endereços são escritos na memória principal?

\item Qual o valor armazenado na tag do endereço \verb|0xBCDE|?

\item Para uma cache de tamanho fixo, qual a consequência do aumento do tamanho de suas vias?

\item Para uma cache de tamanho fixo, qual a consequência do aumento em sua quantidade de vias?

\item Uma cache que sacrifique quantidade de vias por um maior tamanho de via beneficia qual tipo de localidade?

\item Uma cache que sacrifique tamanho de via por uma maior quantidade de vias beneficia qual tipo de localidade?

\item Como aumentar a localidade espacial do programa a seguir?

\begin{center}
    \begin{minipage}{0.575\textwidth}
        \begin{lstlisting}[frame=single]
int main() {
    int i, j;
    int m[8][8], a[8][8], b[8][8]; 
    
    for(j = 0; j < 8; j ++)
        for(i = 0; i < 8; i ++)
            m[i][j] = a[i][j] + b[i][j];
    
    return 0;
}
        \end{lstlisting}
    \end{minipage}
\end{center}

\item Como aumentar a localidade temporal do programa a seguir?

\begin{center}
    \begin{minipage}{0.35\textwidth}
        \begin{lstlisting}[frame=single]
int main() {
    int i = 0, j = 0;
    int a = 0, b = 0;
    
    while(i < 10) {
        j++;
        a += i;
        b = j + 2;
        i++;
        j = b % 2;
    }
}
        \end{lstlisting}
    \end{minipage}
\end{center}

\item Quanto à estratégia Write-Back:
\begin{enumerate}
    \item em que esta consiste?
    \item qual parte do sistema faz seu controle?
    \item como esta interfere na complexidade do hardware implementado?
    \item como esta interfere na miss-penalty?
    \item como esta interfere na velocidade das escritas feitas pelo processador?
\end{enumerate}

\item Quanto à estratégia Write-Through:
\begin{enumerate}
    \item em que esta consiste?
    \item qual parte do sistema faz seu controle?
    \item como esta interfere na complexidade do hardware implementado?
    \item como esta interfere na miss-penalty?
    \item como esta interfere na velocidade das escritas feitas pelo processador?
\end{enumerate}

\section*{Tema: memória virtual}

Para as questões a seguir, considere um sistema com 4 GB de memória física, 64 GB de memória virtual e páginas de 8 kB.

\item Quantos bits são destinados ao offset de página?

\item Quantos bits são destinados ao número de página físico?

\item Quantos bits são destinados ao número de página virtual?

\item Quantas páginas tem a tabela de páginas?

\item Em bits, qual o tamanho da tabela de páginas?

\item Quais componentes do sistema gerenciam a memória virtual?

\item Em um computador executando quatro processos, quantas tabelas de página estão instanciadas? Quantos registradores de tabela de página estão carregados?

\item Quando o estado da memória é refletido na área de swap no disco?

\item Descreva a sequência de acesso a um endereço registrado no TLB.

\item Descreva a sequência de acesso a um endereço não registrado no TLB, mas mapeado na memória principal.

\item Descreva a sequência de acesso a um endereço não registrado no TLB, nem mapeado na memória principal.

\section*{Tema: entrada e saída endereçada por memória}

\item Qual instrução envia dados a uma I/O mapeada em memória?

\item Qual instrução lê dados de uma I/O mapeada em memória?

\item Há alguma informação que indica ao processador com I/O mapeado em memória se este está acessando a memória ou um periférico? Se sim, qual?

\item Como o processador pode acessar I/O por instruções destinadas a acesso à memória?

\item Suponha um periférico com as seguintes características:
\begin{itemize}
    \item registradores:
    \begin{table}[H]
    \centering
    \begin{tabular}{|l|l|l|l|}
        \hline
        nome    & endereço   & modo    & descrição                                                                                            \\ \hline
        ctrl    & 0xFFFFF000 & escrita & \begin{tabular}[c]{@{}l@{}}0: reconhece status\\ 1: inicia leitura\end{tabular}                      \\ \hline
        status  & 0xFFFFF004 & leitura & \begin{tabular}[c]{@{}l@{}}0: aguardando\\ 1: realizando leitura\\ 2: leitura concluída\end{tabular} \\ \hline
        leitura & 0xFFFFF008 & leitura & contém o resultado da leitura                                                                        \\ \hline
    \end{tabular}
    \end{table}

    \item sua leitura retorna valores que devem ser interpretados em ponto fixo;

    \item é inicializado automaticamente, estando pronto para executar quando o programa começa;

    \item ao ser inicializado, está no estado \texttt{aguardando} e o valor em \texttt{leitura} não é válido;

    \item quando uma leitura é inicializada, o valor em \texttt{leitura} só é válido após sua conclusão;

    \item quando uma leitura é concluída, a próxima só é iniciada após \texttt{ctrl} receber \texttt{0}, indicando que o usuário reconhece a sinalização de conclusão;
\end{itemize}
escreva um código MIPS que computa a média de cinco leituras deste periférico.

\section*{Tema: barramentos}

\item Compare um barramento multiplexado entre endereço e dado com um par de barramentos dedicados.

\item É possível multiplexar, também, os sinais de controle? Por quê?

\item Como é dada a organização temporal dos sinais em um barramento síncrono? Como os protocolos verificam a validade das transmissões neste tipo de barramento?

\item Como é dada a organização temporal dos sinais em um barramento síncrono? Como os protocolos verificam a validade das transmissões neste tipo de barramento?

\item Qual o papel do bus-master em um barramento compartilhado?

\item Qual o papel do bus-arbiter em um barramento compartilhado?

\item Em um barramento no qual um mestre deve ter a solicitação atendida imediatamente, qual deve ser a estratégia de alocação implementada?

\end{enumerate}
\end{document}
